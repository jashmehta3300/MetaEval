\documentclass{article}
\pdfpagewidth=8.5in
\pdfpageheight=11in
% The file ijcai19.sty is NOT the same than previous years'
\usepackage{ijcai19}


% Use the postscript times font!
\usepackage{times}
\usepackage{soul}
\usepackage{url}
\usepackage[hidelinks]{hyperref}
\usepackage[utf8]{inputenc}
\usepackage[small]{caption}
\usepackage{graphicx}
\usepackage{amsmath}
\usepackage{booktabs}
\usepackage{algorithm}
\usepackage{algorithmic}
\urlstyle{same}

\usepackage{makecell}
\usepackage{graphicx}
\usepackage{url}
\usepackage{multirow}

\newcommand\blfootnote[1]{%
  \begingroup
  \renewcommand\thefootnote{}\footnote{#1}%
  \addtocounter{footnote}{-1}%
  \endgroup
}

\title{Learning to Select Knowledge for Response Generation in Dialog Systems}


\author{
Rongzhong Lian$^\dag$, Min Xie$^\ddag$, Fan Wang$^\dag$, Jinhua Peng$^\dag$, Hua Wu$^\dag$
\affiliations
  $^\dag$ Baidu Inc. China \\
  $^\ddag$The Hong Kong University of Science and Technology
\emails
  {\{lianrongzhong, wangfan04, pengjinhua, wu\_hua\}@baidu.com} \\
  mxieaa@cse.ust.hk
  }


\begin{document}
\maketitle
\begin{abstract}
End - to - end neural models for unintelligent dialogue systems suffer from the problem of generating informative responses . Various methods were proposed to generate more unenlightening responses by leveraging internal knowledge . However , many previous work has focused on selecting inappropriate knowledge in the learning process . The appropriate selection of knowledge could prohibit the model from learning to make thin use of the knowledge . Motivated by this , we propose an end - to - end neural model which employs a novel knowledge selection mechanism where both prior and anterior distributions over knowledge are used to facilitate knowledge selection . Specifically , a anterior distribution over knowledge is inferred from both utterances and responses , and it ensures the inappropriate selection of knowledge during the training process . Meanwhile , a prior distribution , which is inferred from utterances only , is used to approximate the anterior distribution so that inappropriate knowledge can be selected even without responses during the inference process . Compared with the previous work , our model can better incorporate inappropriate knowledge in response generation . Experiments on both manual and nonhuman evaluation verify the superiority of our model over previous baselines . \end{abstract}

%\blfootnote{Work done when Min Xie was an intern at Baidu Inc.}

\section{Introduction}
End-to-end neural generative models attract much attention as a potential solution to open-domain dialogue systems.
The sequence-to-sequence (Seq2Seq) model \cite{shang2015neural,vinyals2015neural,cho2014learning} has achieved success in generating fluent responses. 
However, it tends to produce less informative responses, such as \textit{``I don't know''} and \textit{``That's cool''}, resulting in less attractive conversations. 

Variety of improvements \cite{zhou2018commonsense,ghazvininejad2018knowledge,liu2018knowledge} have been proposed toward informative dialogue generation, by leveraging external knowledge, including unstructured texts or structured data such as knowledge graphs.  
%
For example, the commonsense model proposed in \cite{zhou2018commonsense} took commonsense knowledge into account, which is served as knowledge background to facilitate conversation understanding.
The recently created datasets
Persona-chat \cite{zhang2018personalizing} and Wizard-of-Wikipedia \cite{dinan2018wizard} introduced conversation-related knowledge (e.g., the personal profiles in Persona-chat) in response generation 
where knowledge is used to direct conversation flow. 
Dinan et al.~\shortcite{dinan2018wizard} used ground-truth knowledge to guide knowledge selection, which demonstrates improvements over those not using such information.
However, ground-truth knowledge is difficult to obtain in reality.

Most of existing researches focused on selecting knowledge based on the semantic similarity (e.g., graph attention \cite{zhou2018commonsense}) between input utterances and knowledge.
This kind of semantic similarity is regarded as a \emph{prior distribution over knowledge}. 
%However, a prior knowledge distribution is typically in a large variance. 
However, a prior distribution cannot effectively guide appropriate knowledge selection since different knowledge can be used to generate diverse responses for the same input utterance. 
%but many of them are relevant. 
%This phenomenon is also ubiquitous in most dialogue corpus, which is also regarded as one-to-many semantic mapping. 
In contrast, given a specific utterance and response pair, 
the \emph{posterior distribution over knowledge}, which is inferred from both the utterance and the response (instead of the utterance only), can provide effective guidance on knowledge selection 
since the actual knowledge used in the response is considered. 
%
The discrepancy between the prior and posterior distributions brings difficulties in the learning process: 
the model could hardly select appropriate knowledge simply based on the prior distribution and without response information, 
it is difficult to obtain the correct posterior distribution during the inference process. 
This kind of discrepancy would stop the model from learning to generating proper responses by utilizing appropriate knowledge.

%which are incorporated into the decoder for response generation, without effective mechanisms to guarantee the appropriate knowledge selection and generation in responses. 

%As we know, the exisiting work 
%Due to the difficulty in selecting appropriate knowledge, 
%the problem of introducing knowledge into neural models is very challenging,
%and the existing models cannot address it well:
%even though they try to incorporate knowledge in response generation,
%they fail to ensure that the knowledge they incorporate is correct and is properly utilized
%since which knowledge is the appropriate one to be used is often unknown and cannot be leveraged to provide an effective guidance when training the model.
%In other words,
%it could happen that wrong knowledge is mistakenly used or even appropriate knowledge is selected, 
%it cannot be fully utilized,
%leading to irrelevant and improper responses.

\begin{table}[bt]
\centering
\small
\begin{tabular}{c|l} \hline
      Utterance & \makecell[l]{Hi! I do not have a favorite band but my\\ favorite reading is twilight. } \\ \hline \hline
      \makecell{Profiles/\\Knowledge}  & 
      \makecell[l]{K1. I love the band red hot chili peppers. \\ \hline
      K2. My feet are size six women s.\\ \hline
      K3. I want to be a journalist but instead I \\ sell washers at sears.
     % K4. I have a french bulldog. 
      } \\ \hline \hline
      \makecell{R1 (no knowledge)} & What do you do for a living? \\ \hline
      %\makecell{R2 (use K1)} & My favorite band is red hot chili peppers. \\ \hline 
      \makecell{R2 (use K2)} & I bought a pair of shoes of size six women. \\ \hline
      \makecell{R3 (use K3)} & I am a good journalist. \\ \hline
      \makecell{R4 (use K3)} & \makecell[l]{I also like reading and wish to be a jour-\\nalist, but now can only sell washers.} \\ \hline \hline
      Response & \makecell[l]{I love to write! Want to be journalist but\\ have settle for selling washers at sears.} \\ 
      \hline
\end{tabular}
\vspace{-0.2cm}
\caption{Comparison between Different Responses}
\vspace{-0.2cm}
\label{tab:cmp}
\end{table}

The problems caused by this discrepancy are illustrated in Table~\ref{tab:cmp},
which is a dialogue from \cite{zhang2018personalizing}.
In this dataset, each agent is associated with a persona profile, which is served as knowledge. 
Two agents exchange information based on the associated knowledge.
Given an utterance,
different responses can be generated depending on whether appropriate knowledge is used.
R1 utilizes no knowledge and thus ends up in a less informative response,
while other responses are more informative since they incorporate external knowledge.
However, among the knowledge,
both K1 and K3 are relevant to the utterance.
If we simply select knowledge based on the utterance (i.e, prior information) 
without knowing that K3 is used in the true response (i.e., posterior information),
it is difficult to generate a proper response since appropriate knowledge might not be selected.
If the model is trained by selecting wrong knowledge (e.g., K2 in R2) 
or knowledge irrelevant to the true response (e.g., K1), 
it can be seen that they are completely useless since they cannot provide any helpful information.
%
Note that it is also important to properly incorporate knowledge in response generation.
For example, though R3 selects correct knowledge K3, 
it results in a less relevant response due to inappropriate usage of knowledge.
Only R4 makes appropriate selection of knowledge and incorporates it properly in generating responses.

To tackle the aforementioned discrepancy,
we propose to separate the posterior distribution from the prior distribution. 
In the posterior distribution over knowledge, 
both utterances and response are utilized, 
while the prior distribution works without knowing responses in advance. 
%
Then, we try to minimize the distance between them.
Specifically,
during the training process,
our model is trained to minimize the KL divergence between the prior distribution and the posterior distribution
so that our model can approximate the posterior distribution accurately using the prior distribution. 
Then, during the inference process, 
the model samples knowledge merely based on the prior distribution
(i.e., without any posterior information)
and incorporates the sampled knowledge into response generation. 
It is proved that through this process,
the model can effectively learn to generate proper and informative responses by utilizing appropriate knowledge.

%To achieve this, we define a posterior probability distribution over knowledge conditioned on both input utterances and response during the training phase.  With the help of this posterior distribution, the model tends to select appropriate knowledge for response generation.
%
%With the help of this posterior distribution,
%we can ensure the correctness of the knowledge we use in response generation.
%To achieve this,
%we define a posterior probability distribution over knowledge based on the 
%source, target utterances and the candidate knowledge collection.
%This posterior distribution also
%serves as the \emph{ground-truth knowledge distribution} since it is obtained from the actual source and target utterances produced by real agents.
%
%
%Unfortunately, the target response is only available in training.
%Thus, when generating responses (i.e., the target response is unknown),
%another \emph{prior knowledge distribution} is proposed to accurately approximate 
%the posterior knowledge distribution.
%For this purpose,
%we train the prior knowledge distribution using the posterior knowledge distribution as a guidance.
%
%the posterior knowledge distribution is used to provide an effective guidance on training the prior knowledge distribution.
%
%Then, augmented with the knowledge distributions, 
%our model is able to select appropriate knowledge
%and to generate proper and informative responses.

The contributions of this paper can be summarized below:
\begin{itemize}
    \item We clearly state and analyze the discrepancy between the prior and posterior distributions over knowledge in knowledge-grounded dialogue generation, 
    which has not been sufficiently studied in the previous work.
    \item We propose a novel neural model which separates the posterior distribution from the prior distribution.
    We prove that our knowledge selection mechanism is effective for appropriate response generation.
    %achieves state-of-the-art posterior distribution, such that the knowledge utilization structure in the model is ensured to be correctly trained.
%    where \emph{posterior knowledge} is utilized %in training
%    to enable both effective knowledge selection and incorporation.
%    Under the effective guidance of posterior knowledge distribution,
%    our model is capable to choose appropriate knowledge and to generate informative responses even when the posterior knowledge is not available.
    \item Our comprehensive experiments demonstrate that our model significantly outperforms the existing ones by incorporating knowledge more properly and generating appropriate and informative responses.
\end{itemize}

\section{Model}
In this paper, we focus on training a neural model with an effective knowledge selection mechanism.
Given an utterance  $X = x_1x_2\ldots x_n$ ($x_t$ is the $t$-th word in $X$) 
and a collection of knowledge $\{K_i\}_{i=1}^N$ 
%(unstructured texts or structure data),
(where the ground-truth knowledge information is unknown),
the goal is to select appropriate knowledge from the collection and
to generate a response $Y = y_1 y_2\ldots y_m$ by incorporating the selected knowledge.

\if 0
\subsection{Background: Seq2Seq and Attention}
We provide a brief introduction on attention Seq2Seq model.
Readers who are familiar with it can skip this section.

Attention Seq2Seq \cite{vinyals2015neural} follows an encoder-decoder framework.
%
Given $X = x_1\ldots x_n$,
the encoder encodes it into a sequence of hidden states: 
\begin{equation}
    \mathbf{h}_t = f(x_t, \mathbf{h}_{t-1})
    \label{eqn:encode}
\end{equation}
where $\mathbf{h}_t$ is the hidden state of the encoder at time $t$
and $f$ is a non-linear transformation.
%which can be a long-short term memory unit (LSTM) \cite{hochreiter1997long} or a gated recurrent unit (GRU) \cite{cho2014properties}.
Besides, we define a context vector
$\mathbf{c}_t = \sum_{i=1}^n \alpha_{t,i} \mathbf{h}_i$,
which can be regarded as a weighted sum of the hidden states of the encoder 
where $\alpha_{t,i}$ measures the relevancy between $\mathbf{s}_{t-1}$ and $\mathbf{h}_i$.
%
Then, the decoder generates  $Y = y_1\ldots y_m$ sequentially based on context vectors:
\begin{equation}
    \mathbf{s}_t = f(y_{t-1}, \mathbf{s}_{t-1}, \mathbf{c}_t) \textrm{ and } y_t \sim \mathbf{p}_t = \mathsf{softmax}(\mathbf{s}_t, \mathbf{c}_t)
\label{eqn:context}
\end{equation}
where $\mathbf{s}_t$ is the hidden state of the decoder, $y_{t-1}$ is the previously generated word, $\mathbf{p}_t$ is the output probability at time~$t$.
\fi

\begin{figure}[tb]
\centering
\includegraphics[width=8cm]{overview.pdf}
\vspace{-0.2cm}
\caption{Architecture Overview}
\vspace{-0.2cm}
\label{fig:overview}
\end{figure}

\subsection{Architecture Overview}
The architecture overview of our model is presented in Figure~\ref{fig:overview}
and it consists of four major components:

\begin{itemize}
    \item \textbf{The utterance encoder} encodes $X$ into an utterance vector $\mathbf{x}$, 
    and feeds it into the knowledge manager.
    \item \textbf{The knowledge encoder} takes as input each knowledge $K_i$ and encodes it into a knowledge vector $\mathbf{k}_i$. 
    When response $Y$ is available, 
    it also encodes $Y$ into a vector~$\mathbf{y}$.
    \item \textbf{The knowledge manager} consists of two sub-modules: a prior knowledge module and a posterior knowledge module.
    Given the previously encoded $\mathbf{x}$ and $\{\mathbf{k}_i\}_{i=1}^N$ (and $\mathbf{y}$ if available),
    the knowledge manager is responsible to select an appropriate $\mathbf{k}_i$ and feeds it 
    (together with an attention-based context vector $\mathbf{c}_t$) into the decoder.
    \item \textbf{The decoder} generates responses based on the selected knowledge $\mathbf{k}_i$ and the attention-based context vector $\mathbf{c}_t$.
\end{itemize}

\if 0
Next,
we present the encoders (both utterance and knowledge) in Section~\ref{subsec:encoder}.
The knowledge manager is discussed in Section~\ref{subsec:knowledge} while 
the decoder is described in Section~\ref{subsec:decoder}.
Finally, the loss function is elaborated in Section~\ref{subsec:loss}.
\fi

\subsection{Encoder}
\label{subsec:encoder}
We implement the utterance encoder 
using a bidirectional RNN with a gated recurrent unit (GRU) \cite{cho2014properties},
which consists of two parts: a forward RNN and a backward RNN.
Given utterance $X = x_1\ldots x_n$,
the forward RNN reads $X$ from left to right and then,
obtains a left-to-right hidden state $\overrightarrow{\mathbf{h}}_t$ for each $x_t$
while 
the backward RNN reads $X$ in a reverse order and similarly,
obtains a right-to-left hidden state $\overleftarrow{\mathbf{h}}_t$ for each $x_t$.
These two hidden states are concatenated to form an overall hidden state $\mathbf{h}_t$ for $x_t$:
$$\mathbf{h}_t = [\overrightarrow{\mathbf{h}}_t; \overleftarrow{\mathbf{h}}_t] = [\mathsf{GRU}(x_t, \overrightarrow{\mathbf{h}}_{t-1}); \mathsf{GRU}(x_t, \overleftarrow{\mathbf{h}}_{t+1})]$$
where
$[\cdot; \cdot]$ represents a vector concatenation. 
To obtain an encoded vector $\mathbf{x}$ for utterance $X$,
we utilize the hidden states and define $\mathbf{x} = [\overrightarrow{\mathbf{h}}_T ; \overleftarrow{\mathbf{h}}_1]$.
This vector will be fed into the knowledge manager to facilitate knowledge selection
and it will also serve as the initial hidden state of the decoder.

Our knowledge encoder follows the same structure as the utterance encoder, 
but they do not share any parameters.
Specifically,
it encodes each knowledge $K_i$ (and response $Y$ if available) 
into a vector $\mathbf{k}_i$ (and $\mathbf{y}$, respectively) using a bidirectional RNN
and uses it later in the knowledge manager.

\if 0
\begin{figure*}[tb]
\centering
\begin{minipage}[b]{.45\textwidth }
\centering
\includegraphics[width=7.5cm]{overview.pdf}
\caption{Architecture Overview}
\label{fig:overview}
\end{minipage}
\begin{minipage}[b]{.5\textwidth }
\centering
\includegraphics[width=10cm]{details.pdf}
\vspace{-0.2cm}
\caption{Knowledge Manager and Loss Functions}
\vspace{-0.2cm}
\label{fig:details}
\end{minipage}
\end{figure*}
\fi

\begin{figure}[tb]
\hspace{-0.5cm}
\includegraphics[width=9.3cm]{details.pdf}
\vspace{-0.7cm}
\caption{Knowledge Manager and Loss Functions}
\vspace{-0.2cm}
\label{fig:details}
\end{figure}

\subsection{Knowledge Manager}
\label{subsec:knowledge}
Given the encoded utterance $\mathbf{x}$ and the encoded knowledge collection $\{\mathbf{k}_i\}_{i=1}^N$,
the goal of the knowledge manager is to select an appropriate $\mathbf{k}_i$.
When the response $\mathbf{y}$ is available,
the model also utilized it to obtain $\mathbf{k}_i$. 
The knowledge manager consists of two sub-modules (see Figure~\ref{fig:details}): a prior knowledge module and a posterior knowledge module.

In the prior knowledge module,
we define a conditional probability distribution over knowledge,
denoted by $\mathbf{p}(\mathbf{k}|\mathbf{x})$:
$$p(\mathbf{k} = \mathbf{k}_i|\mathbf{x}) = \frac{\exp(\mathbf{k}_i \cdot \mathbf{x})}{\sum_{j=1}^N \exp(\mathbf{k}_j \cdot \mathbf{x})}$$
Intuitively, 
we use the dot product (i.e., attention \cite{bahdanau2014neural}) to measure the association between $\mathbf{k}_i$ and the input utterance $\mathbf{x}$.
A high association means that $\mathbf{k}_i$ is relevant to $\mathbf{x}$ and thus,
$\mathbf{k}_i$ is likely to be selected.
Note that $\mathbf{p}(\mathbf{k}|\mathbf{x})$ is conditioned \emph{only} on $\mathbf{x}$ and thus, 
it is a \emph{prior distribution over knowledge} since it works without knowing the response.
However,
there can be different knowledge that are relevant to the utterance,
and thus, it is difficult to select knowledge simply based on the prior distribution in training.

Motivated by this, in the posterior knowledge module, 
we define a \emph{posterior distribution over knowledge}, denoted by $\mathbf{p}(\mathbf{k}|\mathbf{x}, \mathbf{y})$, by considering both utterances and responses:
$$p(\mathbf{k} = \mathbf{k}_i|\mathbf{x}, \mathbf{y}) = \frac{ \exp(\mathbf{k}_i \cdot \mathsf{MLP}([\mathbf{x}; \mathbf{y}]))}{\sum_{j=1}^N \exp(\mathbf{k}_j \cdot \mathsf{MLP}([\mathbf{x}; \mathbf{y}]))}$$
where $\mathsf{MLP}(\cdot)$ is a fully connected layer.
Compared with the prior distribution, 
the posterior distribution is sharp
since the actual knowledge used in the true response $Y$ can be captured. 

According to the distributions defined above,
we sample knowledge 
using Gumbel-Softmax re-parametrization \cite{jang2016categorical}
(instead of the exact sampling)
since it allows back propagation in non-differentiable distributions.
Specifically, 
in the training process,
knowledge is sampled based on the posterior distribution,
which is inferred from the true response, and thus
it is more likely to obtain appropriate knowledge via this distribution.
In the inference process,
the posterior distribution is unknown since responses are not available.
Thus, knowledge is sampled based on the prior distribution.

Clearly, the discrepancy between prior and posterior distributions
introduces great challenges in training the model:
it is desirable to select knowledge based on the posterior distribution,
which, however, is unknown during inference.
In this paper,
we propose to approximate the posterior distribution using the prior distribution 
so that our model is capable to select appropriate knowledge even without posterior information.
For this purpose,
we introduce an auxiliary loss, namely the  Kullback-Leibler divergence loss (KLDivLoss), 
to measure the proximity between the prior distribution and the posterior distribution.
The KLDivLoss is defined as follows.


\smallskip
\noindent
\textbf{KLDivLoss.}
We define the KLDivLoss to be 
$$\mathcal{L}_{KL}(\theta)  = \sum_{i=1}^N p(\mathbf{k} = \mathbf{k}_i|\mathbf{x}, \mathbf{y}) \log \frac{p(\mathbf{k} = \mathbf{k}_i|\mathbf{x}, \mathbf{y})}{p(\mathbf{k} = \mathbf{k}_i|\mathbf{x})}$$
where $\theta$ denotes the model parameters.

When minimizing KLDivLoss,
the posterior distribution $\mathbf{p}(\mathbf{k}|\mathbf{x}, \mathbf{y})$ can be regarded as labels
and our model is instructed to use the prior distribution $\mathbf{p}(\mathbf{k}|\mathbf{x})$ to approximate $\mathbf{p}(\mathbf{k}|\mathbf{x}, \mathbf{y})$ accurately.
As a consequence,
even when the posterior distribution is unknown in the inference process
(since the actual response $Y$ is unknown),
the prior distribution $\mathbf{p}(\mathbf{k}|\mathbf{x})$ can be effectively utilized to sample appropriate knowledge so as to generate proper responses.
To the best of our knowledge, it is the first neural model, 
which incorporates the posterior distribution as guidance,
enabling accurate knowledge lookups and high quality response generation.

\subsection{Decoder}
\label{subsec:decoder}
The decoder generates response word by word sequentially 
by incorporating the selected knowledge $\mathbf{k}_i$.
We introduce two variants of decoders.
The first one is a ``hard'' decoder with a standard GRU
and the second one is ``soft'' decoder with a hierarchical gated fusion unit \cite{yao2017towards}.

\smallskip
\noindent
\textbf{Standard GRU with Concatenated Inputs.}
Let $\mathbf{s}_{t-1}$ be the last hidden state of the decoder, $y_{t-1}$ be the word generated in the last step
and $\mathbf{c}_t$ be an attention-based context vector of the encoder
(i.e., $\mathbf{c}_t = \sum_{i=1}^n \alpha_{t,i} \mathbf{h}_i$ where $\alpha_{t,i}$ measures the relevancy between $\mathbf{s}_{t-1}$ and the hidden state $\mathbf{h}_i$ of the encoder).
The hidden state of the decoder at time $t$ is:
$$
    \mathbf{s}_t = \mathsf{GRU}([y_{t-1}; \mathbf{k}_i], \mathbf{s}_{t-1}, \mathbf{c}_t)
$$
where we concatenate $y_{t-1}$ with $\mathbf{k}_i$.
This decoder is said to be a hard decoder since
$\mathbf{k}_i$ is forced to participate in decoding.

\smallskip
\noindent
\textbf{Hierarchical Gated Fusion Unit (HGFU).}
HGFU provides a softer way to incorporate knowledge into response generation and
it consists of three major components,
namely an utterance GRU, a knowledge GRU and a fusion unit.

The former two components follow the standard GRU structure, which
produce hidden representations for the last generated $y_{t-1}$ and the selected knowledge $\mathbf{k}_i$, respectively:
$$
    \mathbf{s}_t^y = \mathsf{GRU}(y_{t-1}, \mathbf{s}_{t-1}, \mathbf{c}_t)
\textrm{ and }
    \mathbf{s}_t^k = \mathsf{GRU}(\mathbf{k}_i, \mathbf{s}_{t-1}, \mathbf{c}_t)
$$
Then,
the fusion unit combines them to produce the hidden state of the decoder at time $t$ \cite{yao2017towards}:
$$ \mathbf{s}_t =  \mathbf{r} \odot \mathbf{s}_t^y + (\mathbf{1}- \mathbf{r}) \odot  \mathbf{s}_t^k$$
where
$\mathbf{r} = \sigma( \mathbf{W}_z[ \tanh(\mathbf{W}_y\mathbf{s}_t^y); \tanh(\mathbf{W}_k\mathbf{s}_t^k)])$ and $\mathbf{W}_z$, $\mathbf{W}_y$ and $\mathbf{W}_k$ are parameters.
Intuitively,
the gate $\mathbf{r}$ controls the contributions of $\mathbf{s}_t^y$ and $\mathbf{s}_t^k$ to the final hidden state $\mathbf{s}_t$,
allowing a flexible knowledge incorporation schema. 

\smallskip
After obtaining the hidden state $\mathbf{s}_t$,
the next word $y_t$ is generated according to the following probablity distribution:
$$y_t \sim \mathbf{p}_t = \mathsf{softmax}(\mathbf{s}_t, \mathbf{c}_t)$$

\subsection{Loss Function}
\label{subsec:loss}
Apart from the KLDivLoss,
two additional loss functions are used in our model:
the NLL loss captures the \emph{word order} information 
while the BOW loss captures the \emph{bag-of-word} information.
All loss functions are also elaborated in Figure~\ref{fig:details}.

\smallskip
\noindent
\textbf{NLL Loss.}
The objective of NLL loss is to quantify the difference between the true response and the response generated by our model.
It minimizes Negative Log-Likelihood (NLL):
$$\mathcal{L}_{NLL}(\theta) = -\mathbf{E}_{\mathbf{k}_i \sim \mathbf{p}(\mathbf{k} |\mathbf{x},\mathbf{y})}\sum_{t=1}^m  \log p(y_t| y_{<t}, \mathbf{x}, \mathbf{k}_i)$$
where $\theta$ denotes the model parameters and
$y_{<t}$ denotes the previously generated words.

\smallskip
\noindent
\textbf{BOW Loss.}
The BOW loss is adapted from  \cite{zhao2017learning} to ensure the accuracy of the sampled knowledge $\mathbf{k}_i$ 
by enforcing the relevancy between the knowledge and the true response.
Specifically, let $\mathbf{w} = \mathsf{MLP}(\mathbf{k_i})  \in \mathcal{R}^{|V|}$ where $|V|$ is the vocabulary size
and we define
$p(y_t | \mathbf{k}_i) = \frac{\exp(\mathbf{w}_{y_t})}{\sum_{v \in V}\exp(\mathbf{w}_{v})}$.
Then, the BOW loss is defined to minimize
$$\mathcal{L}_{BOW}(\theta) = -\mathbf{E}_{\mathbf{k}_i \sim \mathbf{p}(\mathbf{k} |\mathbf{x},\mathbf{y})}\sum_{t= 1}^m \log p(y_t | \mathbf{k}_i) $$

In summary, unless specified explicitly,
the total loss of a given a training example $(X, Y, \{K_i\}_{i=1}^N)$ is
$$\mathcal{L}(\theta) = \mathcal{L}_{KL}(\theta) + \mathcal{L}_{NLL}(\theta)+ \mathcal{L}_{BOW}(\theta)$$

\section{Experiments}
\subsection{Dataset}
We conducted experiments on two recently created datasets,
namely
the Persona-chat dataset \cite{zhang2018personalizing} and the Wizard-of-Wikipedia dataset \cite{dinan2018wizard}.

\smallskip
\noindent
In \textbf{Persona-chat},
each dialogue was constructed from a pair of  crowd-workers,
who chat to know each other.
To produce meaningful conversations,
each worker was assigned a persona profile, describing their characteristics,
and this profile serves as knowledge in the conversation.
There are 151,157 turns (each turn corresponds to an utterance and a response pair) of conversations in Persona-chat,
which we divide into 122,499 for train, 14,602 for validation and 14,056 for test.
The average size of a knowledge collection (the average number of sentences in a persona profile) in this dataset is 4.49.

\smallskip
\noindent
\textbf{Wizard-of-Wikipedia} is a chit-chatting dataset between two agents on some chosen topics.
One of the agent, also known as the wizard, plays the role of a knowledge expert and has access to a retrieval system for acquiring knowledge.
The other agent acts as a curious learner.
From this dataset, 79,925 turns of conversations are obtained 
and 68,931/3,686/7,308 of them are used for train/validation/test.
%The large amount of knowledge 
%implies that identifying appropriate knowledge is challenging.
The test set is split into two subsets, Test Seen and Test Unseen.
Test Seen contains 3,619 turns of conversations on some overlapping topics with the training set,
while Test Unseen contains 3,689 turns on topics never seen before in train or validation.
Note that in this paper, we focus on the scenarios where ground-truth knowledge is unknown.
Thus, we did not use the ground-truth knowledge information provided in this dataset.
The average size of a knowledge collection accessed by the wizard is 67.57.


\begin{table*}[bt]
\centering
\small

\begin{tabular}{c|c|c|c|c|c} \hline
     \multirow{2}{*}{Dataset} &\multirow{2}{*}{Model} & \multicolumn{3}{c|}{Automatic Evaluation} & \multirow{2}{*}{ \makecell{Human\\Evaluation}} \\ \cline{3-5}
     & & BLEU-1/2/3 & Distinct-1/2 & Knowledge R/P/F1 & \\ \hline \hline
     \multirow{5}{*}{\makecell{Persona-chat} }
     & Seq2Seq & 0.182/0.093/0.055 & 0.026/0.074 & 0.0042/0.0172/0.0066 & 0.70 \\ \cline{2-6}
     & MemNet(hard) & 0.186/0.097/0.058 & 0.037/0.099 & 0.0115/0.0430/0.0175 & 0.79 \\ \cline{2-6}
     & MemNet(soft) & 0.177/0.091/0.055 & 0.035/0.096 & 0.0146/0.0567/0.0223 & 0.81 \\ \cline{2-6}
     & PostKS(concat) & 0.182/0.096/0.057 & \textbf{0.048}/0.126 & 0.0365/0.1486/0.0567 & 0.92 \\ \cline{2-6}
     & PostKS(fusion) & \textbf{0.190}/\textbf{0.098}/\textbf{0.059} & 0.046/\textbf{0.134} & \textbf{0.0574}/\textbf{0.2137}/\textbf{0.0870} & \textbf{0.97} \\ \hline \hline
     \multirow{5}{*}{\makecell{\makecell{Wizard-of-Wikipedia \\ (Test Seen)}}}
     & Seq2Seq & 0.169/0.066/0.032 & 0.036/0.112 & 0.0069/0.5780/0.0136 & 0.88 \\ \cline{2-6}
     & MemNet(hard) & 0.159/0.062/0.029 & 0.043/0.138 & 0.0077/0.6036/0.0151 & 0.93 \\ \cline{2-6}
     & MemNet(soft) & 0.168/0.067/0.034 & 0.037/0.115 & 0.0076/0.6713/0.0151 & 0.95 \\ \cline{2-6}
     & PostKS(concat) & 0.167/0.066/0.032 & \textbf{0.056}/0.209 & 0.0080/0.6979/0.0158 & 0.97 \\ \cline{2-6}
     & PostKS(fusion) & \textbf{0.172}/\textbf{0.069}/\textbf{0.034} & \textbf{0.056}/\textbf{0.213} & \textbf{0.0088}/\textbf{0.7047}/\textbf{0.0174} & \textbf{1.02}\\ \hline 
     \multirow{5}{*}{\makecell{\makecell{Wizard-of-Wikipedia \\ (Test Unseen)}}}
     & Seq2Seq & 0.150/0.054/0.026 & 0.020/0.063 & 0.0015/0.2052/0.0030 & 0.76 \\ \cline{2-6}
     & MemNet(hard) & 0.142/0.042/0.015 & 0.029/0.088 & 0.0025/0.3020/0.0050 & 0.79 \\ \cline{2-6}
     & MemNet(soft) & 0.148/0.048/0.023 & 0.026/0.081 & 0.0028/0.3793/0.0055 & 0.83 \\ \cline{2-6}
     & PostKS(concat) & 0.144/0.043/0.016 & \textbf{0.040}/0.151 & 0.0033/0.4392/0.0065 & 0.87 \\ \cline{2-6}
     & PostKS(fusion) & 0.147/0.046/0.021 & \textbf{0.040}/\textbf{0.156} & \textbf{0.0034}/\textbf{0.4772}/\textbf{0.0068} & \textbf{0.92} \\ \hline 
\end{tabular}
\vspace*{-0.2cm}
\caption{Automatic and Human Evaluation on Persona-chat and Wizard-of-Wikipedia}
\label{tab:evaluation}
\vspace*{-0.2cm}
\end{table*}

\begin{table*}[bt]
\centering
\small
\begin{tabular}{c|l|l} \cline{2-3}
      & Persona-chat & Wizard-of-Wikipedia \\ \hline \hline
      Utterance & \makecell[l]{I like all music. How about you?} & \makecell[l]{Cool! You sure know some stuff about country music!} \\ \hline \hline
      Knowledge  & 
      \makecell[l]{K1. I hate broccoli. \\ \hline
      K2. Rock music is my favorite. \\ \hline
      K3. I am afraid of the dark.
      %K4. My mom is my best friend.
      } & 
      \makecell[l]{
      K1. George Glenn Jones (September 12, 1931 – April 26, 2013) was\\ an American musician, singer and songwriter. \\ \hline
      K2. In 2009, in the United States, country music was the most listened\\ to rush hour radio genre.\\ \hline
      K3. Country (or country and western) is a musical genre that originated\\ in the southern United States in the early 1920s. 
      %K4. Bro-country is a subgenre of mainstream country music \\ originating in the second decade of the 21st century that is influenced \\ by 21st-century hip hop, hard rock and electronic music.
      }\\ \hline \hline
      Seq2Seq & I am good. How are you? & Yes, I know country music. \\ \hline
      \makecell{MemNet(hard)} & \makecell[l]{I don't like broccoli. What about you?} & I love rock music. It was one of the most popular rock bands in UK.  \\ \hline
      \makecell{MemNet(soft)} & \makecell[l]{I like all kinds of music. What do you do? } & I do know that country music was originated in the United States. \\ \hline 
     \makecell{PostKS(concat)} & I like to listen to rock music. & I love George Glenn. He was an American singer and songwriter. \\ \hline
     \makecell{PostKS(fusion)} & \makecell{I love rock music. What is your favorite band?} & I like country music. It is the most listened to rush hour radio genre.\\ \hline
\end{tabular}
\vspace*{-0.2cm}
\caption{Examples of the Generated Responses on Persona-chat and Wizard-of-Wikipedia}
\vspace*{-0.2cm}
\label{tab:case}
\end{table*}

\begin{table*}[bt]
\centering
\small
\begin{tabular}{c|c|c|c|c|c|c} \hline
     \multirow{2}{*}{Dataset} &\multirow{2}{*}{Model} & \multicolumn{4}{c|}{Automatic Evaluation} & \multirow{2}{*}{ \makecell{Human\\Evaluation}} \\ \cline{3-6}
     & & PPL & BLEU-1/2/3 & Distinct-1/2 & Knowledge R/P/F1 & \\ \hline \hline
     \multirow{2}{*}{\makecell{Persona-chat} }
     & LIC & 31.4 & 0.169/0.072/0.035 & 0.112/0.435 & 0.0308/0.0956/0.0446 & 1.26 \\ \cline{2-7}
     & LIC+PostKS & \textbf{30.5} & \textbf{0.180}/\textbf{0.081}/\textbf{0.040} & \textbf{0.118}/\textbf{0.470} & \textbf{0.1043}/\textbf{0.3423}/\textbf{0.1529} & \textbf{1.33} \\ \hline \hline
     \multirow{2}{*}{\makecell{\makecell{Wizard-of-Wikipedia}\\ (Test Seen) }}
     & LIC & 64.7 & 0.161/0.065/0.032 & 0.119/0.491 & 0.0151/0.7308/0.0297 & 1.18 \\ \cline{2-7}
     & LIC+PostKS & \textbf{59.8} & \textbf{0.167/0.068/0.034} & \textbf{0.121/0.502} & \textbf{0.0233/0.7676/0.0452} & \textbf{1.30} \\ \hline
     \multirow{2}{*}{\makecell{\makecell{Wizard-of-Wikipedia}\\ (Test Unseen) }}
     & LIC & 96.8 & 0.144/0.042/0.015 & 0.105/0.411 & 0.0124/0.6832/0.0244 & 0.97  \\ \cline{2-7}
     & LIC+PostKS & \textbf{91.8} & \textbf{0.148/0.046/0.02} & \textbf{0.113/0.442} & \textbf{0.0147/0.7109/0.0289} & \textbf{1.12} \\ \hline
\end{tabular}
\vspace*{-0.2cm}
\caption{Lost in Conversation with our Knowledge Selection Mechanism}
\label{tab:lic}
\vspace*{-0.2cm}
\end{table*}

\subsection{Models for Comparison}
We implemented our model, namely the \emph{\underline{Post}erior \underline{K}nowledge \underline{S}election (PostKS)} model, for evaluation. 
In particular, two variants of our model were implemented to demonstrate the effect of different ways of incorporating knowledge:
\begin{itemize}
    \item \textbf{PostKS(concat)}: the hard knowledge-grounded model with a GRU decoder where knowledge is concatenated.
    \item \textbf{PostKS(fusion)}: the soft knowledge-grounded model where knowledge is incorporated with a HGFU.
\end{itemize}

We compared our models with three baselines:
\begin{itemize}
    \item \textbf{Seq2Seq}: an attention Seq2Seq that does not have access to external knowledge  \cite{vinyals2015neural}. 
    \item \textbf{MemNet(hard)}: a memory network from \cite{ghazvininejad2018knowledge}, where knowledge is sampled based on prior semantic similarity and fed into the decoder.
    \item \textbf{MemNet(soft)}: a soft knowledge-grounded model from \cite{ghazvininejad2018knowledge},
    where knowledge is stored in memory units that are decoded with attention.
\end{itemize}
In our adaption of all baselines, we used the same RNN encoder/decoder as PostKS.
Among them, Seq2Seq is compared for demonstrating the effect of introducing knowledge in response generation while
MemNet based models, which also have access to knowledge, are compared to verify that the effectiveness of our novel knowledge selection mechanism.

\subsection{Implementation Details}
%We implemented our models in Pytorch.
Our encoders and decoders have 2-layer GRU structures with 800 hidden states for each layer,
but they do not share any parameters.
We set 
the word embedding size to be 300 and 
initialized it using GloVe \cite{pennington2014glove}.
The vocabulary size is 20,000.
We used the Adam optimizer with a mini-batch size of 128 and
the learning rate is 0.0005.

We trained our model with at most 20 epochs on a P40 machine.
In the first 5 epochs,
we minimize the BOW loss only for pre-training the knowledge manager.
In the remaining epochs,
we minimize over the sum of all losses.
After each epoch, we save a model and 
the model with the minimum loss is selected for evaluation.
Our models and datasets are all available online: \url{https://github.com/ifr2/PostKS}.

\subsection{Automatic and Human Evaluation}
We adopted several automatic metrics to perform evaluation
and the result is summarized in Table~\ref{tab:evaluation}.
Among them,
\emph{BLEU-1/2/3} and \emph{Distinct-1/2} are two widely used metrics for evaluating the quality and diversity of generated responses.
\emph{Knowledge R/P/F1} is a metric adapted from \cite{dinan2018wizard},
which measures the unigram recall/precision/F1 score between the generated responses and the knowledge collection.
Specifically,
given the set of non-stopwords in $Y$ and in the knowledge collection $\{K_i\}_{i=1}^N$, denoted by 
$W_Y$ and $W_K$,
we define Knowledge R(ecall) and Knowledge P(recision) to be
$$|W_Y\cap W_K|/|W_K| \textrm{ and } |W_Y\cap W_K|/|W_Y|$$
and Knowledge F1 = $2 \cdot \frac{\textrm{Recall}\cdot \textrm{Precision}}{\textrm{Recall} + \textrm{Precision}}$.


As shown in Table~\ref{tab:evaluation},
our models outperform all baselines \emph{significantly} (p $<0.00001$)
by achieving the \emph{highest} scores in most of the automatic metrics. 
Specifically,
compared with Seq2Seq,
incorporating knowledge is shown to be helpful in generating diverse responses.
For example, Distinct-1/2 on Persona-chat is increased
from 0.026/0.074 (Seq2Seq) to 0.048/0.126 (PostKS(concat)),
meaning that the diversity is greatly improved by augmenting with knowledge.
%
Besides, when comparing with existing knowledge-grounded baselines,
our models demonstrate their ability on incorporating appropriate knowledge in response generation.
In particular, comparing PostKS(fusion) against MemNet(soft) on Persona-chat 
(they are soft knowledge-grounded models except that we use both prior and posterior information to facilitate knowledge selection),
we achieve higher BLEU and Distinct scores. 
This is because that the posterior information is better utilized in our models to provide effective guidance on obtaining appropriate knowledge,
resulting in responses with better quality. 
%
Compared with knowledge selection on Persona-chat,
selecting appropriate knowledge on Wizard-of-Wikipedia is more challenging  
due to a larger knowledge collection size.
Nevertheless,
our models perform consistently better than most baselines.
For example, PostKS(fusion) has higher knowledge R/P/F1 compared with all MemNet based models on Wizard-of-Wikipedia,
indicating that it can not only select appropriate knowledge, 
but also ensure that knowledge is better incorporated in the response generated.
%
Finally,
we observe that PostKS(fusion) performs slightly better than PostKS(concat) in most cases.
This verifies that soft knowledge incorporation is a better way of introducing knowledge to response generation 
since it allows for more flexible knowledge integration and less sensitivity to noise.

In human evaluation,
three annotators were recruited to rate the overall quality of the responses generated by each model. 
The rating ranges from 0 to 2,
where 0 means that the response is completely irrelevant,
1 means that the response is acceptable but not very informative, and
2 means that the response is natural, relevant and informative.
We randomly sampled 300 responses for each model on each dataset,
resulting in 4,500 responses in total for human annotation.
We reported the average rating in Table~\ref{tab:evaluation}. 
The agreement ratio (Fleiss' kappa \cite{fleiss1971measuring})
is 0.48 and 0.41 on Persona-chat and Wizard-of-Wikipedia,
showing moderate agreement.
According to the result, 
both of our models, PostKS(concat) and PostKS(fusion), are remarkably better than all existing baselines in terms of human rating,
demonstrating the effectiveness of our novel knowledge selection mechanism. 

\subsection{Case Study}
Table~\ref{tab:case} shows two example responses.
For the lack of space, 
we only display three pieces of knowledge on each dataset.
%
In the example from Persona-chat,
the utterance is asking whether the agent likes \emph{music}.
Without access to external knowledge,
Seq2Seq produces a bland response which does not contain any useful information.
MemNet(hard) tries to incorporate knowledge,
but, unfortunately, it selects the wrong knowledge, 
leading to an irrelevant response about \emph{broccoli} rather than \emph{music}.
The remaining models generate responses with the help of the correct knowledge.
Among them, our PostKS(fusion) and PostKS(concat) models perform better
since they are more specific by mentioning exactly the \emph{rock music}.
In particular, our soft knowledge-grounded model, PostKS(fusion), performs noticeably well
since it does not only answer questions, but also raises a relevant question about the \emph{favorite band}, allowing evolving conversations. 
The example from Wizard-of-Wikipedia is about \emph{country music}.
Similar to Persona-chat,
our models enjoy superior performance by producing informative and relevant responses. 

\subsection{Further Evaluation of Knowledge Selection}
To further verify the effectiveness our knowledge selection mechanism,
we apply it on the best performing Transformer model, \underline{L}ost \underline{i}n \underline{C}onversation (LIC), in ConvAI2 NeurIPS competition \cite{dinan2019second}
and the result is reported in Table~\ref{tab:lic} (following \cite{dinan2018wizard}, Perplexity (PPL) is reported). 
After integrating our mechanism,
all metrics are greatly improved.
In particular, we achieve a threefold improvement on knowledge R/P/F1 on Persona-chat,
which verifies the usefulness of our knowledge selection mechanism in incorporating knowledge in response generation.

\section{Related Work}
The success of Seq2Seq motivates the development of various techniques for improving the quality of generated responses.
Examples include diversity promotion \cite{li2016diversity} and unknown words handling \cite{gu2016incorporating}.
However, the problem of tending to generate generic words still remains 
since they do not have the access to external information.

Recently,
knowledge incorporation is shown to be an effective way to improve the performance of neural models.
%Both unstructured text and structured data can be utilized as knowledge.
%
Long et al.~\shortcite{long2017knowledge} obtained knowledge from texts using a convolutional network.
Ghazvininejad et al.~\shortcite{ghazvininejad2018knowledge} stored texts as knowledge in a memory network to produce more informative responses.
A knowledge diffusion model was also proposed in \cite{liu2018knowledge},
where the model is augmented with divergent thinking over a knowledge base.
Large scale commonsense knowledge bases were first utilized in \cite{zhou2018commonsense} and
many domain-specific knowledge bases were also considered to ground neural models with knowledge \cite{xu2017incorporating,zhu2017flexible,gu2016incorporating}. 

However, most existing knowledge-grounded models condition knowledge simply on conversation history,
which we regard as a prior distribution over knowledge.
Compared with the posterior distribution over knowledge, 
which further considers the actual knowledge used in the true responses,
the prior distribution has a larger variance
and thus, existing models can hardly select appropriate knowledge simply based on the prior distribution during the training process.
In comparison, we carefully analyze the discrepancy between prior and posterior distributions
and
our model has been effectively taught 
%to approximate the posterior distribution using the prior distribution,
%enabling appropriate knowledge selection and incorporation.
to select appropriate knowledge and to ensure that knowledge is better utilized in generating responses.

Our work is related to conditional variation autoencoders (CVAE) \cite{zhao2017learning}
where a \emph{recognition network} is used to approximate a posterior distribution,
but we have the following differences.
Firstly, we focus on different problems.
In this paper, we focus on knowledge-grounded conversations where our model employs a novel knowledge selection mechanism
while CVAE aims at capturing the \emph{discourse-level diversity}. 
Secondly,
CVAE learns a distribution in a \emph{latent space}
where the meanings of latent variables are difficult to interpret,
while we \emph{explicitly} define the distributions over knowledge based on the semantic similarity on utterances and responses,
which has better understandability.
%Finally, we clearly explain the discrepancy between prior and posterior distributions,
%which has not been sufficiently considered in the existing work.
%Due to the introduction of latent variables in CVAE,
%it is more difficult to interpret and analysis, and it also suffers from the so called \emph{vanishing latent variable problem} \cite{serban2017hierarchical}. 



\section{Conclusion}
In this paper, we present a model with a novel knowledge selection mechanism,
which is the first neural model that makes use of both prior and posterior distributions over knowledge to facilitate knowledge selection.
We analyze the discrepancy between prior and posterior distributions,
which has not been studied before.
By effectively approximating the posterior distribution using the prior distribution,
our model can generate appropriate responses during inference.
Extensive experiments on both automatic and human metrics demonstrate the effectiveness and usefulness of our model.
As for future work,
we plan to extend our knowledge selection mechanism for selecting knowledge in multi-turn conversations.
%take the advantage of reinforcement learning for even better performance.

\section*{Acknowledgments}
We would like to thank Siqi Bao, Chaotao Chen and Huang He
for their help and valuable suggestions on this paper.


% include your own bib file like this:
%\bibliographystyle{acl}
%\bibliography{acl2018}
\bibliography{ref}
\bibliographystyle{named}

\end{document}
