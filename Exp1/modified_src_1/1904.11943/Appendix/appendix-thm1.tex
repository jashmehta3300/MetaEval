
\section{Proof for Theorem~\ref{thm:quadratic}}\label{sec:appendix-thm-quadratic}

Here, we provide a more detailed proof of Theorem~\ref{thm:quadratic}.

Consider a quadratic objective function $f(w) = (w - w^*)^T A (w - w^*) / 2$ for some $A \in \mathbb{R}^{d \times d}$ and $w^* \in \mathbb{R}^d$ is the optimal solution.
Assume $A \succeq \mu I$, where $\mu > 0$ is the strong convexity parameter of this function.
Suppose that we want to minimize $f$ using SWALP with gradient samples $\nabla \tilde f(w)$ that satisfy $\mathbb{E}[\nabla \tilde f(w)] = \nabla f(w) = A (w - w^*)$.
Suppose that the variance of these samples always satisfies $\mathbb{E}[\| \nabla \tilde f(w) - \nabla f(w) \|_2^2] \le \sigma^2$ for some constant $\sigma$; this is a standard assumption used in the analysis of SGD.
Then we can prove the following:

\begin{customthm}{\ref{thm:quadratic}}
Suppose we run SWALP under the above assumptions with a cycle length $c$ and $0 < \alpha < \frac{1}{2}\|A\|_2$.
The expected squared distance to the optimum of SWALP's output is bounded by
\[{
\mathbb{E}\left[\|\bar{w}_T - w^*\|^2 \right]
\leq
\frac{\|w_0 - w^*\|^2}{\alpha^2\mu^2T^2} + 
\frac{c(\alpha^2\sigma^2 + \frac{\delta^2d}{4})}{\alpha^2\mu^2T}
}\]
\end{customthm}

\begin{proof}
We start by rewriting the iterates of low-precision SGD as
\begin{align*}
    w_{t+1} &= Q_\delta(w_t - \alpha \nabla \tilde{f}(w_t)) \\
    &= w_t - \alpha A(w_t-w^*) + \zeta_t
\end{align*}
where $\zeta_t$, the error term is explicitly defined as
\begin{align*}
    \zeta_t &= Q_\delta(w_t - \alpha \nabla \tilde{f}(w_t)) - w_t + \alpha A(w_t - w^*) \\
    &= Q_\delta(w_t - \alpha \nabla \tilde{f}(w_t)) - w_t + \alpha A(w_t - w^*) + \alpha \nabla \tilde{f}(w_t) - \alpha \nabla \tilde{f}(w_t) \\
    &= \alpha(A(w_t - w^*) - \nabla\tilde{f}(w_t)) + Q_\delta(w_t - \alpha \nabla \tilde{f}(w_t)) - (w_t - \alpha \nabla \tilde{f}(w_t)).
\end{align*}

We know that $\mathbb{E}[\zeta_i] = 0$ since
\begin{align*}
    \mathbb{E}[\zeta_t] &= \mathbb{E}[\alpha(A(w_t - w^*) - \nabla\tilde{f}(w_t)) + Q_\delta(w_t - \alpha \nabla \tilde{f}(w_t)) - (w_t - \alpha \nabla \tilde{f}(w_t))] \\
    &=\alpha \big(A(w_t - w^*) - \mathbb{E}[\nabla\tilde{f}(w_t)]\big) + \big(\mathbb{E}[Q_\delta(w_t - \alpha \nabla \tilde{f}(w_t))] - (w_t - \alpha \nabla \tilde{f}(w_t))\big) \\
    &= 0.
\end{align*}
Moreover, due to independence, we can bound the variance of $\zeta_t$ as
\begin{align*}
    \mathbb{E}\left[ \| \zeta_t \|^2 \right] 
    &= 
    \mathbb{E}\left[ \| \alpha(A(w_t - w^*) - \nabla\tilde{f}(w_t)) \|^2 \right] 
    + 
    \mathbb{E}\left[ \| Q_\delta(w_t - \alpha \nabla \tilde{f}(w_t)) - (w_t - \alpha \nabla \tilde{f}(w_t)) \|^2 \right] \\
    &\leq
    \alpha^2\sigma^2 + \frac{\delta^2 d}{4}
\end{align*}
where $d$ is the dimension of the solution and $\sigma$ is an upper bound for $A(w_t - w^*)$.
Then 
\begin{align*}
    w_{t+1} - w^* &= w_t - w^* - \alpha A(w-w^*) + \zeta_t \\
    &= (I-\alpha A)(w_t - w^*) + \zeta_t
\end{align*}
Expanding this formula we will have
\begin{align*}
    w_t - w^* = (I-\alpha A)^t(w_0-w^*) + \displaystyle\sum_{i=0}^{t-1} (I-\alpha A)^{t-i-1}\zeta_i
\end{align*}
Computing the distance from the average $\bar{w}_T$ to $w^*$, we get
\begin{align*}
    \bar{w}_K - w^* &= \frac{1}{K}\displaystyle\sum_{t=1}^K w_{ct} - w^* \\
    &=\frac{1}{K}\displaystyle\sum_{t=1}^K\bigg(
    (I-\alpha A)^{ct}(w_0-w^*) + \displaystyle\sum_{i=0}^{ct-1} (I-\alpha A)^{ct-i-1}\zeta_i
    \bigg) \\
    &=\frac{1}{K}\bigg(\displaystyle\sum_{t=1}^K(I-\alpha A)^{ct}\bigg)(w_0-w^*) 
    + \frac{1}{K}\displaystyle\sum_{t=1}^K\displaystyle\sum_{i=0}^{ct-1} (I-\alpha A)^{ct-i-1}\zeta_i. \\
\end{align*}
Note that the first term is a constant; let that be $\chi_K$. Now analyzing the variance:
\begin{align*}
\mathbb{E}\left[\|\bar{w}_K - w^*\|^2\right] &= \mathbb{E}\left[
    \left\|\chi_K + \frac{1}{K}\displaystyle\sum_{t=1}^K\displaystyle\sum_{i=0}^{ct-1} (I-\alpha A)^{ct-i-1}\zeta_i \right\|^2
\right] \\
&= \|\chi_K\|^2 + \mathbb{E}\left[
    \left\|\frac{1}{K}\displaystyle\sum_{t=1}^K\displaystyle\sum_{i=0}^{ct-1} (I-\alpha A)^{ct-i-1}\zeta_i \right\|^2
\right]
\\&\hspace{4em}+ 
    \frac{2}{K}\chi_K^T\displaystyle\sum_{t=1}^K\displaystyle\sum_{i=0}^{ct-1} (I-\alpha A)^{ct-i-1}\mathbb{E}[\zeta_i] \\
&= \|\chi_K\|^2 + \mathbb{E}\left[
    \left\|\frac{1}{K}\displaystyle\sum_{t=1}^K\displaystyle\sum_{i=0}^{ct-1} (I-\alpha A)^{ct-i-1}\zeta_i \right\|^2
\right] \\ 
&= \|\chi_K\|^2 + \frac{1}{K^2} \mathbb{E}\left[
    \left\|\sum_{i=0}^{cK-1}\displaystyle\sum_{t=\lfloor i/c \rfloor + 1}^{K} (I-\alpha A)^{ct-i-1}\zeta_i \right\|^2
\right] 
\end{align*}
Now leveraging the fact that the $\zeta_i$ are zero-mean and independent, we get
\begin{align*}
\mathbb{E}\left[\|\bar{w}_K - w^*\|^2\right]
&= \|\chi_K\|^2 + \frac{1}{K^2} \sum_{i=0}^{cK-1} \mathbb{E}\left[
    \left\| \sum_{t=\lfloor i/c \rfloor + 1}^{K} (I-\alpha A)^{ct-i-1}\zeta_i \right\|^2
\right] \\
&\leq \|\chi_K\|^2 + \frac{1}{K^2} \sum_{i=0}^{cK-1} \left\| \sum_{t=\lfloor i/c \rfloor + 1}^{K} (I-\alpha A)^{ct-i-1} \right\|_2^2 \mathbb{E}\left[
    \left\| \zeta_i \right\|^2
\right]  \\
&\leq \|\chi_K\|^2 + \frac{1}{K^2} \sum_{i=0}^{cK-1} \left\| \sum_{j=0}^{\infty} (I-\alpha A)^j \right\|_2^2 \mathbb{E}\left[
    \left\| \zeta_i \right\|^2
\right] 
\end{align*}
where in the final step the inequality holds because every term in the finite sum within the norm also must appear in the infinite sum.
Next, since $0 < \alpha < \|A\|_2/2$, we know the series sum $ \sum_{j=0}^{\infty} (I-\alpha A)^{j}$ will converge, and we have:
\begin{align*}
\left\| \sum_{j=0}^{\infty} (I-\alpha A)^{j} \right\|_2^2
= 
\left\|(I - (I-\alpha A))^{-1}\right\|_2^2 
=
\left\|\frac{1}{\alpha}A^{-1} \right\|_2^2
\leq
\frac{1}{\alpha^2\mu^2}.
\end{align*}

Putting this back we get,
\begin{align*}
\mathbb{E}[||\bar{w}_K - w^*||^2]
&\leq \|\chi_K\|^2 + 
    \frac{1}{K^2\alpha^2\mu^2} \displaystyle\sum_{i=0}^{cK-1}\mathbb{E}[\|\zeta_i\|^2]  \\
&\leq \|\chi_K\|^2 + 
    \frac{c}{K\alpha^2\mu^2} \left(\alpha^2\sigma^2 + \frac{\delta^2d}{4} \right).
\end{align*}

Finally, we analyze the term $\|\chi_K\|^2$, we will get
\begin{align*}
\|\chi_K\|^2 &= \left\|\frac{1}{K}\big(\displaystyle\sum_{t=1}^K(I-\alpha A)^{ct}\big)(w_0-w^*)\right\|^2 \\
&\leq \frac{1}{K^2}\left\|\displaystyle\sum_{t=1}^K(I-\alpha A)^{ct}\right\|_2^2 \  \left\|w_0 -w^*\right\|^2 \\
&\leq \frac{1}{K^2}\left\|\displaystyle\sum_{t=1}^\infty (I-\alpha A)^t\right\|_2^2 \ \left\|w_0 -w^*\right\|^2 \\
&= \frac{1}{K^2}\left\|(I - (I - \alpha A))^{-1}\right\|_2^2 \ \left\|w_0 -w^*\right\|^2 \\
&= \frac{1}{K^2}\left\|\frac{1}{\alpha}A^{-1}\right\|_2^2 \ \left\|w_0 -w^*\right\|^2 \\
&\leq \frac{1}{K^2\alpha^2\mu^2} \ \left\|w_0 -w^*\right\|^2
\end{align*}

Putting this bound together, we get:
\begin{align*}
    \mathbb{E}[||\bar{w}_K - w^*||^2]
&\leq \frac{1}{K^2\alpha^2\mu^2} \ \|w_0 -w^*\|^2 + 
    \frac{c}{K\alpha^2\mu^2} \left(\alpha^2\sigma^2 + \frac{\delta^2d}{4} \right)
\end{align*}
which is what we wanted to show.
\end{proof}
